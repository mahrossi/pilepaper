\documentclass[aps,prb,superscriptaddress,amsmath,amssymb,showpacs,twocolumn]{revtex4}
%\documentstyle{article}
%\usepackage[paperwidth=210mm,paperheight=297mm,centering,hmargin=1.8cm,vmargin=1.8cm]{geometry}
\usepackage{amsmath}
\usepackage{graphicx}
\usepackage{amssymb}
\usepackage[usenames,dvipsnames]{xcolor}
\usepackage{subfigure}
\usepackage{dcolumn}% Align table columns on decimal point
\usepackage{bm}% bold math
\usepackage{ulem}
\usepackage{caption}
\normalem


\bibliographystyle{apsrev}
\makeatletter
\begin{document}
%opening

\title{An \textit{ad hoc} improvement to ring polymer molecular dynamics}
\author{Mariana Rossi}
\affiliation {Physical and Theoretical Chemistry Laboratory, 
University of Oxford, South Parks Road, Oxford OX1 3QZ, UK}
\author{Michele Ceriotti}
%\email{michele.ceriotti@chem.ox.ac.uk}
\affiliation {EPFL}
\author{David E. Manolopoulos}
\affiliation {Physical and Theoretical Chemistry Laboratory, 
University of Oxford, South Parks Road, Oxford OX1 3QZ, UK}


\newcommand{\avg}[1]{\ensuremath{\left<#1\right>}}
\newcommand{\mc}[1]{{\color{blue}#1}}
\newcommand{\mr}[1]{{\color{Plum}#1}}
%\newcommand{\remove}[1]{{\sout{#1}}}
\newcommand{\remove}[1]{}
\newcommand{\oxford}[1]{{\color{blue} #1}}
\newcommand{\berlin}[1]{{\color{red}#1}}
\newcommand{\Tr}{\ensuremath{\operatorname{Tr}}}
\newcommand{\img}{\ensuremath{\mathrm{i}} }
\newcommand{\diff}{\ensuremath{\mathrm{d}} }
\newcommand{\Cqq}{\ensuremath{\tilde{C}_{qq}(t) }}


\begin{abstract} 
Two of the most successful methods that are available for simulating the quantum dynamics of
condensed phase systems are ring polymer molecular dynamics (RPMD) and centroid molecular dynamics (CMD).
Despiste their conceptual differences, practical implementations of these methods differ in just two aspects:
the choice of the Parrinello-Rahman mass matrix and whether or not a thermostat is applied to the internal 
modes of the ring polymer during the dynamics. 
Here we explore a method which is in a sense `half-way' 
between the two approximations: we keep the path integral bead masses equal to the physical particle
masses but attach a path integral Langevin equation (PILE) thermostat to the internal modes of the
ring polymer. The justification for this is that the inclusion of an internal mode thermostat does
not affect any of the wholesome features of RPMD: because of the choice of bead masses, the 
resulting method is still optimum in the short-time limit, and the transition state approximation
to its reaction rate theory remains closely related to the semiclassical instanton approximation
in the deep quantum tunneling regime. In effect, there is a continuous family of methods with
these properties, parameterised by the coupling strength of the PILE thermostat,
\mc{that equally well preserve the short-time limit of RPMD, which depends on the choice of the mass
matrix but is not sensitive to thermostatting of the internal modes. }
Here we explore numerically how the approximation to quantum dynamics depends on this coupling strength, % MR: maybe not?
with a particular emphasis on vibrational spectroscopy.
\mc{We find that a broad range of values around the critical damping chosen
for the original PILE thermostat give very similar results, that give a reasonable
albeit arguably \emph{ad hoc} approximation to quantum effects, while being 
immune to both the resonance problem of RPMD and the curvature and time step problems of CMD.
\sout{We find that the critical damping chosen
for the original PILE thermostat is close to optimum, % MR: maybe we also don't need to say this.
and that the resulting dynamical approximation
is immune to both the resonance problem of RPMD and the curvature problem of CMD.}
}
\end{abstract}

\maketitle

The quantum nature of light nuclei has a very significant impact on the behaviour
of molecules and materials, not only at cryogenic temperatures, but also at room 
temperature and above. 
Well-established techniques exist to perform atomistic simulations that accurately
and rigorously include nuclear quantum effects (NQEs) on static, time-independent
configurational and thermodynamic
properties\cite{feyn-hibb65book,chan-woly81jcp,parr-rahm84jcp,cepe95rmp},
and efforts are concentrated on making them less demanding, with some success \cite{ceri-mano12prl,  SUZUKI-CHIN/TUCKERMAN} 

For what concerns dynamical properties, however, the situation is much less clear.
Exact techniques are extremely complex, and impractical for anything but the simpler
molecules~\cite{????DAVID????}. Approximate techniques are restricted to either quasi-harmonic
perturbative expansions\cite{????} or one of few approximate, semi-classical 
techniques, most notably the linearized semi-classical initial value representation (LSC-IVR),
centroid molecular dynamics (CMD)~\cite{cao-voth93jcp,cao-voth94jcp}, and ring-polymer 
molecular dynamics (RPMD).
LSC-IVR can be derived rigorously starting from quantum mechanical expressions, but 
the approximations that are practically affordable are affected by severe
zero-point energy leakage, which makes it very hard to collect satisfactory statistics
and casts shadows on the applicability to long-time properties of anharmonic systems.
CMD and RPMD both can be regarded as real-time versions of imaginary-time path integrals,
CMD being essentially classical molecular dynamics on the centroid potential of mean force,
RPMD being classical molecular dynamics based on the ring polymer Hamiltonian. 
Both are to an extent arbitrary, \emph{ad hoc} approximations, even though recent 
efforts have put them on somehow more robust grounds \cite{rich-alth09jcp,hele-alth13jcp,jangvoth99jpc, jangvoth99jpc2, MORE?}.
They are only exact for linear operators in the harmonic limit\cite{habe+13arpc, jangvoth99jpc2}, and 
both are known to exhibit artefacts that are most evident when computing vibrational
spectra of molecules~\cite{witt+09jcp, ivanov+10jpc, habe+08jcp}, that are related to the so-called curvature problem for CMD
(that we will interpret as a consequence of adiabatic separation between the centroid
and the internal modes of the ring polymer) and resonance between physical modes
of the system and internal modes of the ring polymer for RPMD.

Here we will discuss the effect of thermostatting the internal modes of the ring
polymer while using the physical ring polymer mass matrix, thereby obtaining a method
that can be regarded as a ``hybrid'' of RPMD and CMD. We will show that both 
the short-time limit accuracy of RPMD and the analogy between RPMD and the instanton
in the study of rates are not affected by the use of stochastic thermostats for the
internal modes of the  ring polymer. This means that there is a whole family of methods,
differing by the details of the thermostatting, that are not more a \emph{ad hoc}
than RPMD, and offer a way to improve upon the existing approximations. 
While we cannot at this stage suggest a way to exploit this additional
freedom to obtain a more rigorous method for quantum dynamics, we show that
for a broad range of thermostat parameters, this stochastic term
cures the resonance problem of RPMD without triggering the curvature
problem. Furthermore, this method can be ued with much larger time steps
than CMD, as it uses the RPMD mass matrix and does not rely on adiabatic separation.

The Langevin RPMD approach we introduce (which we refer to as \textit{PILE}) is a practical solution to explore
the role of NQEs on dynamical properties, even though we cannot claim that there is 
a universal choice of the damping that gives a rigorously better approximation 
to quantum dynamics. It is quite possible that by exploring the additional 
degrees of freedom that are associated with stochastic dynamics of the 
internal modes of the ring polymer future research may eventually fulfill this 
arduous goal.

\section{ALL SECTIONS FROM DAVID \label{sec:invariance}}


%\section{The resonance problem}
%
%\mc{Do we want to put the analytical results for the off-diagonal coupled oscillators here?}
%\mr{If we can show just a picture about how PILE changes the spectrum, it would be nice.}
%
%We here define the ring polymer effective Hamiltonian, in the bead representation, as
%
%\begin{eqnarray}
%H_n & = & \underbrace{\sum_i^N \sum_j^n \left[\frac{\vert {\bf p}_i^{(j)} \vert^2}{2m^{\prime}_i} + \frac{1}{2}m_i\omega_n^2 \vert {\bf q}_i^{(j)} - {\bf q}_i^{(j-1)} \vert^2 \right]}_{H_0} \nonumber \\
%& & + \sum_j V({\bf q}_1^{(j)}, ..., {\bf q}_N^{(j)}), 
%\label{eq:pimd-hamiltonian}
%\end{eqnarray}
%
%\noindent where $H_0$ is the free ring polymer Hamiltonian, $i$ runs through the $N$ particles of the physical system, $j$ runs through the $n$ path integral replicas,
%${\bf p}_i^{(j)}$ and ${\bf q}_i^{(j)}$ are the momenta and positions associated with particle $i$, replica $j$, respectively, $V$ is the physical interacting potential, and $\omega_n=nk_bT/\hbar$. We note that while $m_i$ should correspond to the physical masses of the system, there is no 
%$a priori$ requirement for the choices of $m^{\prime}_i$ in the standard PIMD formulation, since this term is just added to the Halmiltonian
%to allow the evaluation of the dynamics. Finally, ${\bf q}_i^{(n+1)} = {\bf q}_i^{(1)}$, which closes the ring polymer. 
%
%As explained in detail in Ref. \cite{ceri+10jcp}, it is possible to perform a normal mode transformation on the Hamiltonian above, such that
%the free ring polymer Hamiltonian $H_0$ can be written as
%
%\begin{eqnarray}
%H_0 & = &  \frac{1}{2}{\bf \tilde{P}}^T {\bf M}^{-1} {\bf \tilde{P}} + \sum_i^N \sum_{k=0}^{n-1} \left[ \frac{1}{2}m_i\omega_k^2 ({\bf \tilde{q}}_i^{(k)})^2 \right],
%\label{eq:h0-normalmode}
%\end{eqnarray}
%
%\noindent where $\tilde{P}$ is an array containing all normal mode transformed momenta ${\bf \tilde{p}}_i^{(k)}$, ${\bf \tilde{q}}_i^{(k)}$ are the normal mode  transformed coordinates, and $\omega_k = 2 \omega_n \sin(k\pi/n)$. In RPMD, the mass matrix ${\bf M}$ simply a diagonal matrix containing all the physical masses $m_i$ of the system, and the time evolution of position and momenta are carried out without any thermostat attached to them. In the normal mode representation, $k$=0 is the mode connected to the centroid and $k>0$ represent the internal modes of the ring polymer. It is already worth pointing out that $\omega_1 \approx 2\pi/(\beta\hbar)$ for the free ring polymer lies around 1300 cm$^{-1}$ at 300K and around 435 cm$^{-1}$ at 100K (a temperature that will be important later on in this paper).
%
%As pointed out in Refs. \cite{habe+08jcp, witt+09jcp, ceri+11jcp}, if one applies the RPMD formalism to a simple potential corresponding to a chain of uncoupled harmonic oscillators $V({\bf q})=\sum_i m_i\omega_i^2q_i^2/2$,
%it is straightforward to show that the frequencies of vibration of ring polymer will be given by \mr{check!}
%
%\begin{eqnarray}
%\omega_k^{i}=\sqrt{\omega^2_i + 4 \omega_n^2 \sin^2(k\pi/n)},
%\label{eq:harm-freq-rpmd}
%\end{eqnarray}
%
%\noindent such that the centroid vibrates at the frequencies of the physical system and the other internal modes vibrate at higher frequencies. Clearly, in any real system, the potential will not be exactly harmonic, and the expression in Eq. \ref{eq:harm-freq-rpmd} will be just an approximation for the internal frequencies of the ring polymer. Nevertheless, Eq. \ref{eq:harm-freq-rpmd} already shows clearly the origin of the so-called ``resonance-problem" of RPMD, thoroughly discussed in Refs. \cite{habe+08jcp, witt+09jcp}: For systems with vibrational frequencies $\omega_i$ spanning from small to large wavenumbers, there will be an $\omega_k$ for, e.g. a small $\omega_i$ that will have a very similar frequency to another larger $\omega_i$, so that they will resonate. The lower the temperature, the more severely will these spurious frequencies contaminate the true spectrum, so that for most real-life applications, RPMD cannot be used for the evaluation of vibrational spectra.
%
%%\mr{Show stupid plots of eq. \ref{eq:harm-freq-rpmd} for 300K and 100K with varying $\omega$ that I added to the "extra" folder?}
%
%\section{The curvature problem}
%
%The resonance problem is characteristic to RPMD. CMD avoids contamination of the 
%spectrum by internal modes of the ring polymer by (partial) adiabatic decoupling,
%that ensures that there is no overlap between the range of frequencies of the 
%centroid vibration and that of the non-zero frequency internal modes.
%On the contrary, the so-called curvature problem is characteristic of CMD, 
%and consists in a red shift of stretching modes of groups of atoms that also 
%possess librations or wagging modes, that is accompanied by a broadening of 
%the peak and that becomes more and more pronounced as the temperature decreases. 
%
%As discussed in Ref.~\cite{witt+09jcp}, the curvature problem can be understood
%as arising because the centroid moves on an effective potential, in which 
%the stretching mode is averaged over a soft, strongly non-linear 
%motion of the molecule. In fact, the curvature problem is most severe for a freely-rotating
%bond, and becomes less dramatic when the spread of the ring polymer orthogonal to
%the bond is restricted by angular restraints~\cite{witt+09jcp}, or by the presence 
%of weak interactions with other molecules~\cite{paesani-voth10jcp}.
%
%Note that one could regard this effect as a spurious coupling between
%the physical, centroid component of the stretching mode and the 
%(adiabatically decoupled) internal vibrations of the ring polymer
%in the perpendicular, free-rotation motion. There is in this regards
%an interesting connection with the resonance problem of RPMD: at 
%least in the case of the simple models discussed in Ref.~\cite{witt+09jcp}, 
%resonances happen when the \emph{free ring-polymer} internal mode frequencies 
%match the frequency of the stretching, indicating that the resonance problem
%is also exacerbated by the presence of soft, strongly non-linear wagging modes.
%
%Since the curvature problem is caused by the centroid moving on a mean 
%potential surface, averaged over the internal modes, one could 
%wonder whether thermostatting the internal modes of the ring polymer
%as we advocate here would be enough to cause a similar effect, even 
%without changing the mass matrix.
%
%
%\section{The ``PILE'' method}
%
%Since it was shown in Section \ref{sec:invariance} that the application of the PILE
%thermostat on the internal modes of the ring polymer does 
%not disturb the short time limit of RPMD \mr{(still to be shown)}, we here apply this
%technique in order to obtain vibrational spectra free of the resonances that 
%affect RPMD. We call this method \textit{PILE} in the following.
%
%As explained in Ref. \cite{ceri+10jcp}, applying the PILE thermostat amounts to
%the following equations of motion in the normal mode representation,
%
%\begin{eqnarray}
%\frac{d}{dt}{\tilde{\bf q}}_i^{(k)} & = & {\tilde{\bf p}}_i^{(k)}/m_i, \nonumber \\
%\frac{d}{dt}{\tilde{\bf p}}_i^{(k)} & = & -m_i\omega_k^2{\tilde{\bf q}}_i^{(k)} - \gamma^{(k)} {\tilde{\bf p}}_i^{(k)} + \sqrt{\frac{2m_i\gamma^{(k)}}{\beta_n}} {\boldsymbol\xi}_i^{(k)}(t),
%\end{eqnarray}
%
%\noindent where ${\boldsymbol\xi}_i^{(k)}$ is a vector of Gaussian-distributed, uncorrelated random noise. 
%The $\gamma^{(k)}$ friction coefficients that yield critical damping are the ones that minimize
%the correlation time $\tau_H$ for the Hamiltonian of the harmonic oscillator, subject to the Langevin equations of
%motion above. As explained in Ref. \cite{ceri+10jcp}, the analytical result is,
%
%\begin{equation}
%\tau_H=\frac{1}{\gamma^{(k)}} + \frac{\gamma^{(k)}}{4\omega_k^2},
%\end{equation} 
%
%\noindent so that the optimum value is $\gamma^{(k)}_c=2\omega_k$. The difference of the treatment here,
%as compared to the one in Ref. \cite{ceri+10jcp} is that we do set $\gamma^0=0$, i.e., the centroid mode
%is not thermostatted and follows Hamiltonian dynamics. 
%
%As shown in Section \ref{sec:invariance}, we do not find a criterion to fix the $\gamma^{(k)}$ parameters.
%While choosing them to yield critical damping seems a sensible choice, we also investigate the effect of
%changing these parameters to yield underdamped and overdamped regimes of the Langevin equation in the
%following sections.
%
%\section{The quartic oscillator}
%
%
%In this section, we show the results we obtain when applying RPMD, CMD, and the \textit{PILE} method 
%to the one dimensional anharmonic quartic potential $V(q)=q^4/4$, already explored previously in the literature (e.g. Refs. \cite{crai-mano04jcp, jangvoth99jpc}).
%We here consider $\hbar=1$, $m=1$, and $\beta=10$, in natural units. For the RPMD and \textit{PILE} simulations, we used 64 beads, and a time step $\Delta t = 0.01$ for the integrator, which ensures convergence and accuracy of all relevant quantities in the simulation. For CMD, we used a time step of 0.002 natural units.
%\mr{[comment: change PILE for \textit{PILE} in the following?]}
%
%\begin{figure}[htbp]
%\begin{center}
%\includegraphics[width=0.48\textwidth]{figures/vv_autocorr_quartic}
%\end{center}
%\caption{Velocity autocorrelation function for the quartic one-dimensional potential $V(q)=q^4/4$. The dotted line corresponds to the exact result. All units are atomic units, $m$=1, and $\beta$=10.}
%\label{fig:autocorr-quartic}
%\end{figure}
%
%In Figure \ref{fig:autocorr-quartic}, we show the velocity autocorrelation functions calculated from 10000 trajectories of 200 time units. 
%All correlation functions were obtained after thermalization runs with a simple Andersen thermostat. We plot the correlation function of RPMD, CMD,
%PILE ($\gamma^{(k)}=\gamma^{(k)}_c$), underdamped PILE ($\gamma^{(k)}=0.1\gamma^{(k)}_c$), and overdamped PILE ($\gamma^{(k)}=10\gamma^{(k)}_c$). The dotted lines correspond to the
%exact result, in which the autocorrelation function does not decay. All methods predict an artificial decorrelation of the autocorellation function, with CMD being the closest to the exact result and PILE (critical damping), being the second best. The fact that critically damped PILE gives a better result
%than RPMD and than its underdamped and overdamped variations is not expected to be always true - it just happens to be the
%case for this potential. In any case, what we do expect, given the mathematical derivations above, is that PILE still gives a good approximation
%to the real quantum result in a wide range of systems.
%
%\begin{figure}[htbp]
%\begin{center}
%\includegraphics[width=0.38\textwidth]{figures/vv_fourier_quartic.pdf}
%\end{center}
%\caption{Fourier transform of the velocity autocorrelation function for the quartic one-dimensional potential $V(q)=q^4/4$. The dotted line corresponds to the expected exact result for the position of the peak. All units are atomic units, $m$=1, and $\beta$=10.}
%\label{fig:spec-quartic}
%\end{figure}
%
%
%We note that for this 1D potential there will be no resonance problem in RPMD in the main peak, since the ring-polymer frequencies are always shifted to higher frequencies with respect to the first physical one ($\approx$ Eq. \ref{eq:harm-freq-rpmd}), and there is only one frequencies. We see this effect in the overtones but since they have very low intensity, they do not affect the spectrum. Also, there is no curvature problem, simply because it is a 1D potential. In Figure 
%\ref{fig:spec-quartic} the corresponding Fourier transforms of the autocorrelation functions shown in Fig. \ref{fig:autocorr-quartic} are shown, which just confirm the considerations above. CMD and critically damped PILE produce the narrowest peaks, with the exact result being a $\delta$-function centered at the position of the dotted line. 

\section{Solving the resonance and curvature problems}

In the previous section we have demonstrated that adding a stochastic thermostat to the internal degrees of freedom
of a ring polymer without changing the mass matrix preserves all of the wholesome properties of RPMD, and only reduces
the short-time limit accuracy by one order in time. In Ref.~\cite{witt+09jcp} it was shown that both RPMD and CMD 
exhibits clearly unphysical artefacts when it comes to modeling infra-red spectroscopy and vibrational dynamics. 
Namely, vibrational spectra obtained by RPMD are contaminated by resonances between the physical vibrational modes
of the system and the internal modes of the ring polymer, whyle CMD spectra are affected by a curvature problem that
causes peaks corresponding to high frequency vibrations to broaden and shift to low frequency when the temperature
decreases. 
Note that both of these problems can be traced to the coupling between high-frequency modes and 
near-zero frequency rotations or wagging modes of the covalent bonds. Resonances are observed when \emph{free ring polymer}
frequencies overlap with the range of frequencies corresponding to physical oscillations of the system. 
In the presence of a harmonic potential of frequency $\omega$, the ring polymer frequencies are shifted to
$\sqrt{\omega^2 + \tilde{\omega}^2_k}$, and so in a truly one-dimensional system there can never be a resonance. 


\subsection{Model molecules: OH and CH$_4$}


\begin{figure}[htbp]
\centering
\includegraphics[width=0.5\textwidth]{figures/comparison_oh_factors.pdf}
\caption{Dipole absorption cross section for the OH model molecule: (a) and (b) CMD and PILE methods at 100, 200, and 300K ; (c)  and (d) RPMD and PILE methods at 109, 450, and 436 K. The dotted grey line corresponds to the classical vibrational frequency predicted for this model.}
\label{fig:oh-rpmd-cmd-pile}
\end{figure}


In this section we study two model molecules, already treated in Ref. \cite{witt+09jcp}.
They are the OH and the CH$_4$ molecules, for which the interatomic potential is
given by

\begin{equation}
\phi=\sum_{bonds} \frac{k_b}{2} (r-r_{eq})^2 + \sum_{angles} \frac{k_a}{2}(\theta - \theta_{eq})^2,
\end{equation}

These model molecules are especially interesting because they are known cases where
both the curvature and the resonance problems have been well studied by Witt and coworkers\cite{witt+09jcp}.
As in Ref. \cite{witt+09jcp}, we chose $k_b=0.49536$ Ha/bohr$^2$ and $r_{eq}=1.0$\AA~ for the OH molecule,
and  $k_b=0.30345$ Ha/bohr$^2$, $k_a=3.1068\times10^{-5}$ Ha/deg$^2$, $r_{eq}=1.09$\AA, and $\theta_{eq}=107.8$deg for
the CH$_4$ molecule. For these models, we calculate the dipole absorption cross section given by 


\begin{figure}[htbp]
\centering
\includegraphics[width=0.4\textwidth]{figures/oh_rpmdvspiledampings_109K.pdf}
\caption{Dipole absorption cross section for the OH model molecule at 109K. The various plots show the RPMD spectrum compared to the PILE methods with different $\gamma^{(k)}$ parameters.}
\label{fig:oh-rpmd-pile-dampings}
\end{figure}

\begin{equation}
n(\omega)\alpha(\omega)= \frac{\pi \beta \omega^2}{3cV\epsilon_0} \tilde{I}(\omega),
\end{equation}

\noindent where $\tilde{I}(\omega)$ is the Fourier transform of the Kubo-transformed dipole autocorrelation function.
It is this autocorrelation function that we approximate with CMD, RPMD, and the PILE method. We here also filter out
translations and rotations of the molecule using the scheme proposed in Ref. \cite{witt+09jcp}.


For the OH molecule we calculate the RPMD spectra at the same temperatures and number of beads discussed in Ref. \cite{witt+09jcp}, 
namely 109 K and 64 beads (resonant case), 350K and 16 beads (non-resonant case), and 436K and 16 beads (resonant case). We used a time
step of 0.25 fs for these simulations.
In Figure \ref{fig:oh-rpmd-cmd-pile}(c) and (d), we compare these spectra with the corresponding PILE spectra (critical damping).
The classical frequency of vibration for this molecule and the parameters chosen here
would lie at 3715.6 cm$^{-1}$ \mr{check!}.
As in Ref.\cite{witt+09jcp}. the vibrational frequencies of the ring polymer in the RPMD method 
split the peak at the resonant temperatures. The PILE spectra yield a 
single peak, which is at the same place at all temperatures and slightly
red-shifted with respect to the classical vibrational frequency. Also, the non-resonant frequencies
of the ring polymer cannot be observed in the PILE spectra. We also notice that the PILE
spectrum gets broader as the temperature is lowered. This effect arises from the fact that at lower
temperatures there are more internal frequencies of the ring polymer which interact with the physical frequency ... \mr{[comment: must discuss!]}.

\begin{figure}[htbp]
\centering
\includegraphics[width=0.5\textwidth]{figures/comparison_ch4_factors.pdf}
\caption{Dipole absorption cross section for the CH$_4$ model molecule: (a) and (b) CMD and PILE methods at 100, 200, and 300K; (c) and (d) RPMD and PILE methods at 136, 273, and 450 K.}
\label{fig:ch4-rpmd-cmd-pile}
\end{figure}

In Fig. \ref{fig:oh-rpmd-cmd-pile}(a) and (b) we compare the CMD and PILE spectra for the OH molecule
at 100K, 200K, and 300K. The CMD simulations were run with a time step of 0.01 fs, while the PILE simulations
were run with a time step of 0.25 fs. We used 64 beads at 100K, and 16 beads at 200 and 300K.
The curvature problem is clear for the CMD spectra: at 100K the peak is red-shifted to as low as 2200 cm$^{-1}$
and massively broadened. Even at 300K some of the curvature problem is still observed. The PILE method
does not show this effect. Although the PILE peak broadens slightly as the temperature
lowers, the peak stays at the same position for all temperatures. The severe curvature problem seen in CMD is thus
closely related to the choice of the dynamic mass matrix.


In Figure \ref{fig:oh-rpmd-pile-dampings} we compare different values of the $\gamma^{(k)}$ parameters in the PILE method,
both in the underdamped and overdamped regimes for the IR spectrum of the OH molecule at 109K. The position
and width of the peaks are affected by changing this parameter, but not substantially - similar to what
we observe for the quartic potential. Going as high as ten times the critical damping, or as low
as one tenth of it, does not change the peak position by more than 3\% of its wavenumber. Since we do not
find a mathematical criterion to fix these parameters, this uncertainty will be intrinsic to this method.

For the CH$_4$ molecule we calculate the same data as we did for the OH molecule. We use all
the same simulation settings as for the OH molecule. The results are reported
in Figure \ref{fig:ch4-rpmd-cmd-pile}. In Fig. \ref{fig:ch4-rpmd-cmd-pile}(c) and (d), it is clear that the resonance
problem of RPMD is healed by the PILE method for the double resonant case (136K) and single resonant case (273K). 
In Fig. \ref{fig:ch4-rpmd-cmd-pile}(a) and (b), it is also shown that the PILE method predicts
the bending peak ($\approx$1200 cm$^{-1}$) at the same position as CMD, while the stretching peak
at high frequencies does not show the curvature problem in the PILE method. 
For the higher temperatures (300K and 400K),
where both the curvature problem of CMD and the resonance problem of RPMD are not severe, both spectra
agree very well with the PILE one. Overall, we again observe a broadening 
of the stretching peak in the PILE method as the temperature is lowered. All observations are consistent with the OH case.

Next, we investigate a non-empirical potential energy surface in order to see
how these methods compare.


\section{Real OH}

We consider the real OH molecule with the interatomic potential given by a Morse-type potential of the form

\begin{equation}
\phi = D_{e} \left[ 1-e^{-\alpha(r-r_{e})}\right]^2 
\end{equation}

\noindent where $D_{e} = \omega_e^2/4\omega_e\chi_e$, $\alpha=\sqrt{2\mu_{OH}\omega_e\chi_e}$. The parameters $\omega_e$=3737.76 cm$^{-1}$, $\omega_e \chi_e$=84.881 cm$^{-1}$, and $r_{e}$ = 0.96966 \AA\, were obtained from the parameters measured for the real OH molecule, as reported in Ref. [HERZBERG].

\begin{figure}[htbp]
\centering
\includegraphics[width=0.5\textwidth]{figures/comparison_ohanharm_factors.pdf}
\caption{Dipole absorption cross section for the real OH molecule: (a) and (b) CMD and PILE methods at 100, 200, and 300K ; (c)  and (d) RPMD and PILE methods at 109, 450, and 436 K. The dotted grey line corresponds to the anharmonic (exact) vibrational frequency predicted for this model.}
\label{fig:oh-rpmd-cmd-pile}
\end{figure}


\section{Zundel cation}

\begin{figure}[htbp]
\begin{center}
\includegraphics[width=0.48\textwidth]{figures/zundel_cleanrot_cmd_rpmd_pile.pdf}
\end{center}
{\small {}$^a$ Ref. \cite{YehLee1989}; {}$^b$ Ref. \cite{AsmisScience2003}; {}$^c$ Ref. \cite{VendrellMeyer2007}}
\caption{(a) Reference data for the IR spectrum of H$_5$O$_2^+$ taken from the literature: experimental
IRMPD data for the OH-stretch region at 300K from Ref. \cite{YehLee1989}, experimental IRMPD data between 650 and 1900 cm$^{-1}$ 
at 100K from Ref. \cite{AsmisScience2003}, and theoretical multi configuration time dependent Hartree (MCTDH) spectrum at 0K from Ref. \cite{VendrellMeyer2007}.
IR spectrum of H$_5$O$_2^+$ obtained from the Fourier transform of the dipole autocorrelation from molecular 
dynamics trajectories in the CCSD(T) parameterised surface of Ref. \cite{HuangBraamsBowman2005}, at $T$=100K: 
(b) Comparison between CMD and classical spectra, where the curvature problem is clear especially for the OH stretch peaks; 
(c) Comparison between RPMD and classical spectra, where the resonances can be seen at all peaks above 1500 cm$^{-1}$; (c)
Comparison between PILE and classical spectra, where both of the problems mentioned in (a) and (b) are not present, but the peaks
are much broader. In (d), substantial shifts due to NQE in the evaluation of the spectra can be identified.}
\label{fig:zundel-spectra}
\end{figure}


We apply the methods discussed so far
to a real and accurate potential energy surface of 
a real molecule, namely, the Zundel cation H$_5$O$_2^+$.
The IR spectrum of  H$_5$O$_2^+$ has been extensively studied in the literature~\cite{Schatteburg2008}, both theoretically 
\cite{AgostiniCiccotti2011, ParkKim2007, VenerSauer2001, SauerDoebler2005, ChengKrause1997, VendrellMeyer2007, 
BaerMarxMathias2010, KaledinBowmanJordan2009, HuangBraamsBowman2005} 
and experimentally~\cite{YehLee1989, GuascoJohnson2011, AsmisScience2003, HammerBowmanCarter2005, FridgenMaitre2004}. 
Particular attention has been given to a doublet structure in the 
shared proton stretch region that can be measured
at low temperatures~\cite{GuascoJohnson2011, HammerBowmanCarter2005}.
Theoretically,  this structure can only be captured
by certain levels of theory that include anharmonic effects and nuclear quantum effects, 
like the multi-configutarional time-dependent Hartree (MCTDH) method at
0K \cite{VendrellMeyer2007}. It is completely absent from any harmonic treatment,
and also expected to be absent from the methods used here due to the lack
of quantum phase information in any of them.
This molecule presents an excellent testing ground for new methodologies
also because of the availability of a potential energy and dipole surfaces parametrized
with CCSD(T) data that was published by Huang, Braams and Bowman~\cite{HuangBraamsBowman2005},
which is both very accurate and inexpensive to evaluate. 

For the Zundel cation, we evaluate the classical, CMD, RPMD, and PILE IR spectra at 100K.
We chose this temperature because it is a temperature low enough to highlight the issues of
RPMD and CMD, and because actual experiments are performed at this temperature\cite{AsmisScience2003}.
In Fig. \ref{fig:zundel-spectra}, these spectra are shown in panels (b), (c), and (d). In panel (a)
we show reference data from the literature, both experimental (Refs. \cite{YehLee1989,AsmisScience2003}) and a high level theoretical
spectrum using the MCTDH method (Ref. \cite{VendrellMeyer2007}). 

It is clear from
Fig. \ref{fig:zundel-spectra}(b)
that the curvature problem of CMD is massive for the OH stretch peaks for this
real potential energy surface. For RPMD, the resonant frequencies of the
ring polymer split both the OH stretch peaks at high wavenumbers and the
shared proton bending mode at around 1700 cm$^{-1}$. The PILE method,
here only reported for critical damping, successfully heals both problems,
although especially at higher wavenumbers, the peaks seem to be considerably broadened.

None of the approximate methods investigated here can reproduce the
doublet peak at around 1000 cm$^{-1}$, which is shown in Fig. \ref{fig:zundel-spectra}(a).
Given that this doublet structure has been interpreted as a Fermi resonance
involving a fourth-order coupling between the proton transfer mode,
the O-O stretching mode, and the H-O-H wagging mode \cite{VendrellMeyer2007, Schatteburg2008},
it does not come as a surprise that it is not present in the methods investigated here:
None of these methods contain the necessary physics to describe such a resonance.
However, only extremely computationally expensive methods [CITE?] are able to describe
such effects, which also become less important as the system size grows. In the realm
of approximate methods for including nuclear quantum effects for dynamical observables,
and that can be applied for systems of high dimensionality without an extreme
computational cost, the newly proposed PILE method
thus emerges as the most promising candidate. 



\section{Conclusions}








\bibliography{biblio}
\end{document}
